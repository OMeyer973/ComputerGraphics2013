\documentclass{article}
\usepackage{amsmath, amssymb, mdwlist, graphicx, hyperref}


\newcommand{\mpar}[1]{\marginpar{\textit{#1}}}
\newcommand{\norm}[1]{\Vert #1 \Vert}
\DeclareMathOperator{\argmax}{argmax}
\DeclareMathOperator{\rank}{rank}
\DeclareMathOperator{\argmin}{argmin}
\newenvironment{solution}{\paragraph{Solution.}$\,$ }{\vskip 3mm\hrule}
\newenvironment{exercise}[2]{\begin{verse}\textbf{Exercise #1 (#2pt).} }{
\end{verse}\medskip}
\newcommand{\bbR}{\mathbb{R}}
\newcommand{\bw}{\mathbf{w}}
\newcommand{\bx}{\mathbf{x}}
\newcommand{\bd}{\mathbf{d}}
\newcommand{\bb}{\mathbf{b}}
\newcommand{\bs}{\mathbf{s}}
\newcommand{\bn}{\mathbf{n}}
\newcommand{\bF}{\mathbf{F}}
\newcommand{\ba}{\mathbf{a}}
\newcommand{\bc}{\mathbf{c}}
\newcommand{\bq}{\mathbf{q}}
\newcommand{\be}{\mathbf{e}}
\newcommand{\dd}{\mathrm{d}}
\newcommand{\pd}[2]{\frac{\partial #1}{\partial #2}}
\newcommand{\br}{\mathbf{r}}
\newcommand{\by}{\mathbf{y}}
\newcommand{\bzero}{\mathbf{0}}
\newcommand{\bz}{\mathbf{z}}
\newcommand{\bSigma}{\mathbf{\Sigma}}
\newcommand{\bp}{\mathbf{p}}
\newcommand{\bm}{\mathbf{m}}
\newcommand{\bM}{\mathbf{M}}
\newcommand{\bK}{\mathbf{K}}
\newcommand{\bD}{\mathbf{D}}
\newcommand{\bA}{\mathbf{A}}
\newcommand{\bX}{\mathbf{X}}
\newcommand{\bY}{\mathbf{Y}}
\newcommand{\bR}{\mathbf{R}}
\newcommand{\bI}{\mathbf{I}}
\newcommand{\bS}{\mathbf{S}}
\newcommand{\bT}{\mathbf{T}}
\newcommand{\balpha}{\boldsymbol{\alpha}}
\newcommand{\pt}[2]{\left(\begin{array}{c}#1\\#2\end{array}\right)}



\begin{document}
\title{MTAT.03.015 Computer Graphics (Fall 2013)\\
Lecture V: Math exercises}
\author{Konstantin Tretyakov\\
\medskip
\\
\parbox{9cm}{\small Solution for every task gives 0.5 points. Solutions are accepted on paper or via e-mail (\texttt{kt@ut.ee}) until October 16, 2013.}
}
\date{}
\maketitle


\begin{enumerate}

\item Let the horizontal field of view (\emph{fov-X}) of some \emph{view-frustum} be 75 degrees. Let the screen dimensions be $1280\times 1024$. Find the corresponding vertical field of view  (\emph{fov-Y}).

\item Consider a perspective projection in two-dimensional space. We shall be projecting to the line $y = 1$ with $(0, 0)$ as the center of projection.
\begin{itemize}
	\item Find the projection matrix in homogeneous coordinates.
	\item Explain what linear transformation does this matrix correspond to in the three-dimensional homogeneous space. Illustrations are welcome.
\end{itemize}

\item Let $ax + by + cz + d = 0$ be some plane in three-dimensional space and let $P = (p_x, p_y, p_z)$ be a point not located on this plane. Find a matrix, that performs a perspective projection from $P$ onto this plane (in homogeneous coordinates).

\item Let $P_1 = (x_1, y_1, z_1)$, $P_2 = (x_2, y_2, z_2)$ -- be points in space. Consider some attribute $\mathcal{A}$ (e.g. color) assigned to the points. Suppose that point $P_1$ is assigned attribute value $a_1$, point $P_2$ --- value $a_2$ and on the line between them the attribute varies linearly.

Let $P_1^*$, $P_2^*$ --- be the perspective projections of points $P_1$ and $P_2$ onto the plane $z = z_n$ with $(0, 0, 0)$ as the center of projection.
Let $P_t^*$ be a point obtained by interpolating between $P_1^*$ and $P_2^*$: 
$$P_t^* = tP_1^* + (1-t)P_2^*,$$
and let $P_t = (x_t, y_t, z_t)$ be the point of the segment $[P_1, P_2]$ that projects into $P_t^*$. Show that the value $a_t$ of the attribute at point $P_t$ satisfies
\[ \frac{a_t}{z_t} = t \frac{a_1}{z_1} + (1-t)\frac{a_2}{z_2}\, .\]
Try to find a simple geometric proof to this fact.

It follows from this result, than when you are rasterizing a triangle, which was obtained via perspective projection, you cannot simply interpolate attribute values (e.g. colors or texture coordinates) along the screen as you did in the practice session\footnote{\url{http://en.wikipedia.org/wiki/Texture_mapping\#Perspective_correctness}}.

\end{enumerate}
\end{document}
